\cleardoublepage
\chapternonum{缩略词表}
\begin{center}
    % uncomment line 5 if you want to set the fontsize of the table.
    % \zihao{-4}
    % comment line 8 and uncomment line 9 if you need to set the table as mid-aligment.
    % or addjust the paras as your personal need.
    \begin{longtable}{m{2cm}m{8cm}m{5cm}}
    % \begin{longtable}{m{2cm}<{\centering}m{8cm}<{\centering}m{5cm}<{\centering}} 
        \toprule
        \textbf{英文缩写}&\textbf{英文全称}&\textbf{中文全称}\\
        \midrule
        \endfirsthead
        \toprule
        \textbf{英文缩写}&\textbf{英文全称}&\textbf{中文全称}\\
        \midrule
        \endhead 
        \bottomrule
        \endfoot
        \bottomrule
        \endlastfoot
        MBE&Maximal Biclique Enumeration&极大二分团枚举\\
        SE tree&Set Enumeration Tree&集合枚举树\\
        DFS&Depth-First Search&深度优先搜索\\
        ASE tree&Aggressive Set Enumeration Tree&主动的集合枚举树\\
        AMP&Aggressive Merge-based Pruning&主动的顶点合并剪枝\\
        PMP&Passive Merge-based Pruning&被动的顶点合并剪枝\\
        AMBEA&Aggressive Maximal Biclique Enumeration Algorithm&主动的极大二分团枚举算法\\
        LCG&Local Computational subGraph&局部计算子图\\
        CG&Computational subGraph&计算子图\\
        BDS&Bitmap-based Dynamic Subgraph&基于位图的动态子图方法\\
        AdaMBE&Adaptive Maximal Biclique Enumeration & 自适应的极大二分团枚举\\
        INF&Infinity&无穷大,表示时间超出限制\\        
        SM&Streaming Multiprocessor & 流处理器\\
        CSR&Compressed Sparse Row& 压缩稀疏行格式\\
        GMBE&GPU-based Maximal Biclique Enumeration & 基于GPU的极大二分团枚举\\
        

	\end{longtable}
\end{center}