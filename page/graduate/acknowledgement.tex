\cleardoublepage
\chapternonum{致谢}

求是园韶光十载,如白驹过隙,历历在目。蓦然回首,恍如隔世,百感交集。在此,感谢一路陪伴、给予我帮助的老师、同窗以及家人朋友们。

% 首先,我要感谢我的博士生导师何水兵研究员。在科研方面,何老师的追求卓越的科研精神深深地影响着我。从我刚入学开始,何老师就对我的科研成果进行严格的把控,将我的每一个科研成果都撰写成了CCF A类会议论文。在与顶级会议论文审稿人的十余次切磋交流中,我不断成长和进步,坚定了科研的信念。在论文截止日期前,何老师对我论文的字斟句酌,如今回想起来,仍旧记忆犹新。在为人处世方面,何老师中庸的处事态度也让我受益匪浅。在人生的道路上,常常会遇到各种荆棘和困难,何老师总是教导我要像司马懿一样,安心积累成果,等待机遇,不要心急。除此之外,我还要感谢何老师给予我参与课程助教、撰写项目和基金申请书等机会,这让我提前全面了解了科研的基本套路,为我日后从事科研相关行业积累了宝贵的经验。

首先,感谢我的博士生导师何水兵研究员。在科研方面,您追求卓越、严谨认真的科研态度深深地影响着我。从我刚入学开始,您就对我的科研成果进行严格的把控,将我的每一个科研成果都按照计算机领域的最高标准,撰写成了CCF A类会议论文。在与顶级会议论文审稿人的切磋交流中,我不断成长和进步,坚定了科研的信念。每当论文截稿时间临近时,您总是起早贪黑,像打磨每一件艺术品一样,对论文字斟句酌,如今回想起来,仍旧记忆犹新。在为人处事方面,您更是我的人生导师。您时常向我们现身说法,分享您几十年科研生活的经历与经验,分析其中的利弊得失。您告诉我们人生是一场马拉松,走科研路要耐得住寂寞,安心积累成果,等待机遇,不要心急。每当遇到困恼的时候,您的谆谆教诲总会在耳边回响,引导我走向正确的方向。感谢您,让我博士阶段的研究生活充实而有意义。

其次,感谢我的硕士生导师姜晓红副教授。您在我本科学习期间的计算机体系结构课程,不仅激发了我对这一领域的深厚兴趣,更引领我走上了科研之路。在硕士学习期间,您全方位关注我的成长,积极为我搭建科研实习的桥梁,并多次提供国内外学术交流的宝贵机会。您还鼓励并亲自带领我们参加FPGA设计大赛,为我提供了宝贵的实践和竞技舞台。同时,您的信任与支持,让我有机会带领本科生开展SRTP项目,这对我的科研任务完成起到了重要的辅助作用。感谢您,为我硕士阶段的研究生活增添了丰富的色彩。

再次,感谢研究生生涯中帮助过我的老师和同学们。感谢张学琛、汪睿、孙贤和、银燕龙、王洋、陈刚、曾凯、马德、顾宗华、吴健、杨莹春等老师在论文构思与写作过程中提供了宝贵的指导与建议。感谢李旭、杨斯凌、陈伟剑、陈平、陈帅犇、李振鑫、党政、洪佩怡、胡双、徐耀文、吴桐、段和霄、朱建新、宗威旭、张文捷、宋浩喆、常子汉、詹璇、王文炯、王文涛、刘镕琛、瞿浩阳、陈新宇、张瑞东、郭家豪、廖亦宸、周方、李翔、陈广、杜定益、曹聪、孙浩、聂宗涛、胡志强等同学们的陪伴,让我的研究生生活不再孤单,感谢你们给予我的支持与鼓励,愿你们每个人都能拥有一个光明且 美好的未来。

此外,特别感谢在此期间陪伴着我的家人和朋友们。感谢父母对我长期的默默支持与无私奉献,是你们给予了我读博路上不竭的动力与勇气。感谢姑父在学术探索中给予我的一次次有力的帮助与指导。感谢在这六年硕博生涯中与我保持联系的老友们,是你们的陪伴给我单调的求学生活带来了无尽的温暖与光亮。衷心感谢每一位在我的学术旅程中给予我帮助与支持的人,是你们让我的研究生涯充满了意义与价值。

最后,感谢在百忙之中为我博士论文提出宝贵意见的论文盲审专家与答辩评委,谢谢你们对我论文的严格把关,为我的博士生涯画上圆满的句号。


\begin{flushright}
  潘哲\\
  二〇二四年六月于求是园
\end{flushright}