\begin{algorithm} [H]
  \begin{algorithmic}[1]
    \normalsize

    \STATE $processing\_v$是一个全局变量,初始化为0;
    \STATE $SM\_task\_queue$是用于负载均衡的全局并发队列 ;
     
    \renewcommand{\algorithmicwhile}{\textbf{procedure}}
    \renewcommand{\algorithmicdo}{\textbf{:}}
    

    \WHILE{\textsf{warp\_kernel}}
    \renewcommand{\algorithmicwhile}{\textbf{while}}
    \renewcommand{\algorithmicdo}{\textbf{do}}
      \WHILE{\textbf{true}}
        \IF{$SM\_task\_queue$非空}
          \STATE $(L,R,C) \leftarrow SM\_task\_queue$\textsf{.dequeue}$()$ ;
        \ELSE
          \STATE $v_s$ = \textsf{atomicInc}$(processing\_v)$
          \IF{$v_s \in V$}
            \STATE $L \leftarrow N(v_s); R\leftarrow\{v_s\}; C\leftarrow \emptyset $;
            \FOR{$v_c \in N_2(v_s)$}
              \IF{$L\cap N(v_c) = L$}
                \STATE $R \leftarrow R \cup \{v_c\}$;
              \ELSIF{$v_c$的索引在$v_s$之后}
                \STATE $C \leftarrow C \cup \{v_c\}$;
              \ENDIF
            \ENDFOR
          \ELSE
            \STATE \textbf{return};
          \ENDIF
        \ENDIF

        \IF {$R = \Gamma(L)$}
          \IF {$\min\{|L|,|C|\}\times|C| >$ 数量边界,并且$\min\{|L|,|C|\} >$ 高度边界 }
            \FOR{$v_t \in C$}
              \STATE $L_t \leftarrow N(v_t); R_t\leftarrow R; C_t\leftarrow \emptyset $;
              \FOR{$v_c \in C$}   
                \IF {$L_t \cap N(v_c) = L_t$}
                  \STATE $R_t \leftarrow R_t \cup \{v_c\}$;
                \ELSIF{$L_t \cap N(v_c)$ 非空}
                  \STATE $C_t \leftarrow C_t \cup \{v_c\}$;
                \ENDIF
              \ENDFOR
              \STATE $SM\_task\_queue$\textsf{.enqueue}$((L_t, R_t, C_t))$
              \STATE $C \leftarrow C \setminus \{v_t\}$\;
            \ENDFOR
          \ELSE
            \STATE \textsf{iteratively\_search}$(L,R,C)$;
          \ENDIF

        \ENDIF
      \ENDWHILE
    \ENDWHILE 

  \end{algorithmic}
  \caption{负载感知的基于任务的调度方法}
  \label{alg:gmbe_task_full}
\end{algorithm}