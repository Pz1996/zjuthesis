\chapter{总结与展望}

本章对全文的研究内容和主要创新点做出总结,并对极大二分团枚举问题的未来可以继续探索的研究问题和方向进行展望。

\section{工作总结}

随着信息技术的迅猛发展,大规模数据的生成和积累已成为常态。为了充分挖掘和利用这些数据中的有效关联信息,二分图结构被广泛应用于表示两个不同群体之间的联系,比如在电子商务场景中用户和商品之间的购买关系。极大二分团是二分图中的一类稠密子图,代表着数据集中那些紧密连接的群体,能够有效揭示二分图中存在的某种规律或共同特征,为探究群体行为内部的脉络和联系提供帮助。极大二分团枚举问题的目标是高效地识别并枚举给定二分图中的全部极大二分团,该问题具有广泛的应用场景,并受到学术界和工业界的广泛关注。本文的主要工作是研究高效的极大二分团枚举方法。针对极大二分团枚举问题中搜索空间庞大、使用单一数据结构以及并行扩展性差等问题,本文从搜索空间剪枝方法、自适应的数据结构和基于GPU的并行实现三个方面分别提出了三种独立的极大二分团枚举方法。通过大量实验证明,这三种方法相较于现有的极大二分团枚举方法均表现出明显的性能优势。本文的主要研究内容和贡献如下:

\textbf{(1)激进的极大二分团枚举剪枝方法:} 为了优化极大二分团枚举问题的搜索空间,现有工作设计了多种优化方法。然而,现有工作的搜索空间仍然非常庞大,在枚举过程中仍然会生成大量的非极大二分团,消除非极大二分团的过程带来高昂的节点检查开销。对此,我们提出了激进的极大二分团枚举算法AMBEA,主要包括激进的集合枚举树(ASE)和激进的顶点合并剪枝方法(AMP)两个核心技术点。激进的集合枚举树打破了现有枚举树仅允许使用部分顶点进行节点生成的结构限制,总是使用全部顶点将每个二分团扩展为极大二分团,裁剪了大量产生非极大二分团的分枝。激进的顶点合并剪枝方法打破了现有剪枝方法由特定顶点触发的限制,通过改变节点的生成过程,主动地合并具有相同局部邻居的顶点,提升剪枝效率。实验结果表明,AMBEA对比最新的ooMBEA等算法压缩了2.37-8.98倍的搜索空间,缩短了1.15-5.32倍的运行时间。

\textbf{(2)自适应的极大二分团枚举数据结构:} 现有的极大二分团枚举问题研究往往注重于算法优化,却忽略了数据结构表示所带来的固有低效性问题,导致现有方法仅适用于小数据集,其中包含不超过1亿个极大二分团。针对这一问题,我们提出了自适应的极大二分团枚举算法AdaMBE,主要包括基于位图的动态子图方法(BDS)和局部邻居缓存方法(LNC)两大核心技术。考虑到默认方法在邻接表上进行大量集合运算会带来高计算开销,基于位图的动态子图方法在枚举过程中动态生成位图,并通过位图上的位运算加速集合运算。该方法充分结合了邻接表与位图两种不同表示方法的优势,同时也考虑到二分图中两个顶点集大小不同的特性。为解决默认方法中访问顶点的完整邻居导致的冗余计算问题,局部邻居缓存方法动态保存顶点的部分活跃邻居,避免大量对结果无影响的不活跃邻居访问,从而减少冗余计算。这类冗余计算包括对非极大二分团节点的计算、重复的集合运算以及无效顶点的访问等。实验结果显示,AdaMBE相比最新的ooMBEA等算法,运行时间缩短了1.6至49.7倍,并成功处理了TVTropes数据集中超过190亿个极大二分团的枚举需求。

\textbf{(3)基于GPU的极大二分团枚举并行实现:}目前,现有的极大二分团枚举算法均采用CPU实现,但其并行扩展性受到CPU计算核心数量的限制。为解决这一问题,我们引入了具有大量计算核心的GPU作为计算资源,用于加速极大二分团枚举过程。我们提出了基于GPU的高效极大二分团枚举解决方案GMBE,主要包括基于枚举节点重用的迭代计算流程、基于局部邻居数量感知的剪枝方法以及负载感知的任务调度方法,分别用于解决内存短缺、线程分歧和负载不均等问题。基于枚举节点重用的迭代计算流程仅存储子枚举树根节点以及对应的元数据,通过重用枚举树根节点的内存,避免为新枚举节点动态分配内存,从而减少内存开销。基于局部邻居数量感知的剪枝方法通过记录枚举过程中顶点局部邻居数量的变化来对枚举空间进行剪枝,并通过批量比较局部邻居的方法,在剪枝的同时最小化线程分歧问题。负载感知的任务调度方法将每棵子枚举树对应的计算任务与一个线程束%的计算资源 // 对齐
进行绑定,在运行时动态预估子枚举树的大小并对较大的子枚举树进行进一步拆分,实现了细粒度的负载均衡。实验结果显示,基于GPU的GMBE方法比现有基于96个CPU的并行算法ParMBE实现了70.6倍的性能提升。

综上所述,本文聚焦于极大二分团枚举问题。从剪枝方法、数据结构和并行实现三个方面进行探索,提出了三种独立的解决方案和多个通用的核心技术点。这些方法将极大二分团算法的枚举能力提升至百亿级别,并将其实现介质扩展至GPU。

\section{研究展望}

极大二分团枚举问题作为图数据挖掘和图论领域中的经典难题,在广泛应用的同时,与图模式挖掘、极大团枚举等问题密切相关。尽管本文已经就极大二分团枚举问题展开了多方面的探索和实践,但相关领域仍有许多问题值得深入探究和发掘。未来的研究可以在以下几个方面展开:

\textbf{(1)极大二分团枚举问题深度优化:} 首先,从技术路线的宏观层面来看,目前本文提出的三条技术路线相互独立。对枚举性能的深入挖掘可以从融合不同的技术路线入手。例如,可以将AdaMBE中的基于位图的动态子图方法应用于AMBEA或GMBE算法中,结合邻接表和位图两种图表示方法对这两种算法进行进一步的性能优化。其次,从枚举过程中节点计算的中观层面来看,我们研究发现节点检查在枚举过程中占用大量的枚举时间。虽然目前的研究方法已经提出大量的剪枝优化方法,但高开销的节点检查过程仍无法避免。考虑到现有方法都是针对算法本身设计的启发式方法,未来的研究工作可以尝试引入机器学习方法,通过深度网络模型对枚举节点进行快速批量检查,进而优化枚举性能。最后,从集合运算和顶点访问的微观层面来看,虽然AdaMBE方法注意到了计算过程中的细粒度冗余问题,但冗余计算仍然大量存在。未来的研究工作可以从冗余的角度进行深入优化,例如从系统层面同时处理多个枚举节点并实时动态减少冗余,进一步提升极大二分团枚举的性能。

\textbf{(2)相关图挖掘问题的深度优化:} 在本文的研究过程中,我们发现极大二分团枚举问题中的技术挑战在相关图挖掘问题中同样存在。这导致目前的极大团枚举问题的GPU实现方案性能不如传统的CPU实现,在图模式挖掘问题中,目标子图内的顶点数量通常较少(不超过10个)。为了解决这一问题,未来的研究工作可以考虑以下几个方面的优化:首先,可以将AMBEA中的剪枝优化方法迁移到相关问题中,以突破相关问题中枚举树的结构限制。通过引入更高效的剪枝策略,可以减少枚举过程中的无效计算,从而提升相关图挖掘问题的性能。其次,可以将AdaMBE中混合不同图表示的思路迁移到相关问题中,加速问题求解过程中大量的集合交集运算。通过选择合适的图表示方法,并结合适当的数据结构和算法设计,可以有效降低集合操作的计算复杂度,从而提高相关图挖掘问题的效率。最后,可以将GMBE中枚举节点重用与任务感知的细粒度任务调度方法迁移到相关问题中,进一步拓展图模式挖掘问题中的目标子图规模。通过合理地利用枚举节点重用技术和任务感知的调度策略,可以优化多任务之间的调度效率,提高相关图挖掘问题的可扩展性和并行性。通过以上的优化措施,我们有望在相关图挖掘问题中实现更高效的算法设计与实现,进而推动相关图挖掘领域的进步。

\textbf{(3)极大二分团枚举问题的应用领域扩展:} 近年来,研究者们在不同类型的二分图上定义了各种形式的二分团,从而拓宽了二分图数据挖掘的研究内容和应用领域。考虑到极大二分团枚举问题在这类挖掘中的基础地位,我们的研究工作将在相关衍生问题中扮演着基础性的角色。例如,在最大边二分团搜索问题中,我们可以基于任一极大二分团枚举算法,并结合现有的最大边二分团搜索问题的剪枝策略,提出新的高效实现方案。同样地,我们的研究成果也可应用于优化类似的二分团枚举问题,如相似二分团枚举、(p,q)二分团枚举、公平极大二分团枚举以及二分团渗透社区等问题的求解。这为未来更多自定义二分团挖掘算法的优化提供了多方面的思路与启示。