\cleardoublepage
\chapternonum{致谢}

求是园韶光十载,如白驹过隙,历历在目。蓦然回首,恍如隔世,百感交集。在此,感谢一路陪伴、给予我帮助的老师、同窗以及家人朋友们。

首先,我要感谢我的博士生导师何水兵研究员。在科研方面,何老师的追求卓越的科研精神深深地影响着我。从我刚入学开始,何老师就对我的科研成果进行严格的把控,将我的每一个科研成果都撰写成了CCF A类会议论文。在与顶级会议论文审稿人的十余次切磋交流中,我不断成长和进步,坚定了科研的信念。在论文截止日期前,何老师对我论文的字斟句酌,如今回想起来,仍旧记忆犹新。在为人处世方面,何老师中庸的处事态度也让我受益匪浅。在人生的道路上,常常会遇到各种荆棘和困难,何老师总是教导我要像司马懿一样,安心积累成果,等待机遇,不要心急。除此之外,我还要感谢何老师给予我参与课程助教、撰写项目和基金申请书等机会,这让我提前全面了解了科研的基本套路,为我日后从事科研相关行业积累了宝贵的经验。

其次,我要感谢我的硕士生导师姜晓红副教授。姜老师在我本科学习期间的计算机体系结构课程,不仅激发了我对这一领域的深厚兴趣,更引领我走上了科研之路。在硕士学习期间,姜老师全方位关注我的成长,积极为我搭建科研实习的桥梁,并多次提供国内外学术交流的宝贵机会。姜老师还鼓励并亲自带领我们参加FPGA设计大赛,为我提供了宝贵的实践和竞技舞台。同时,姜老师的信任与支持,让我有机会带领本科生开展SRTP项目,这对我的科研任务完成起到了重要的辅助作用。感谢姜老师为我硕士阶段的研究生活增添了丰富的色彩。

再次,我要感谢研究生生涯中帮助过我的老师和同学们。感谢张学琛、汪睿、孙贤和、银燕龙、王洋、陈刚、曾凯、马德、顾宗华、吴健、杨莹春等老师在论文构思与写作过程中提供了宝贵的指导与建议。感谢李旭、杨斯凌、陈伟剑、陈平、陈帅犇、李振鑫、党政、洪佩怡、胡双、徐耀文、吴桐、段和霄、朱建新、宗威旭、张文捷、宋浩喆、常子汉、詹璇、王文炯、王文涛、刘镕琛、瞿浩阳、陈新宇、张瑞东、郭家豪、廖亦宸、周方、李翔、陈广、杜定益、曹聪、孙浩、聂宗涛、胡志强等同学们的陪伴,让我的研究生生活不再孤单,感谢你们给予我的支持与鼓励,愿你们每个人都能拥有一个光明而美好的未来。

最后,我要向我的家人和朋友表达最深切的谢意。感谢父母对我长期的默默支持与无私奉献,是你们给予了我读博路上不竭的动力与勇气。感谢姑父在学术探索中给予我的一次次有力的帮助与指导。感谢在这六年硕博生涯中与我保持联系的老友们,是你们的陪伴给我单调的求学生活带来了无尽的温暖与光亮。衷心感谢每一位在我的学术旅程中给予我帮助与支持的人,是你们让我的研究生涯充满了意义与价值。



\begin{flushright}
  潘哲\\
  二〇二四年六月于求是园
\end{flushright}