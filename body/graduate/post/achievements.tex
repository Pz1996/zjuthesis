\cleardoublepage
\chapternonum{攻读博士学位期间取得的科研成果}

\section*{已发表的学术论文}

[1] \textbf{Zhe Pan}, Shuibing He, Xu Li, Xuechen Zhang, Yanlong Yin, Rui Wang, Lidan Shou, Mingli Song, Xian-He Sun, Gang Chen. ``Enumeration of Billions of Maximal Bicliques in Bipartite Graphs without Using GPUs''. Proceedings of the International Conference for High Performance Computing, Networking, Storage and Analysis (SC). 2024. Accept. (对应第三章)

[2] \textbf{Zhe Pan}, Shuibing He, Xu Li, Xuechen Zhang, Rui Wang, Gang Chen. ``Efficient Maximal Biclique Enumeration on GPUs". Proceedings of the International Conference for High Performance Computing, Networking, Storage and Analysis (SC). 2023. No. 16, pp. 1-13. (对应第四章)

[3] \textbf{Zhe Pan}, Zonghua Gu, Xiaohong Jiang, Guoquan Zhu, De Ma. ``A Modular Approximation Methodology for Efficient Fixed-Point Hardware Implementation of the Sigmoid Function". IEEE Transactions on Industrial Electronics (TIE). 2022. Vol. 69, No. 10, pp. 10694-10703.

\begin{sloppypar}
[4] \textbf{Zhe Pan}, Yuruo Jin, Xiaohong Jiang, Jian Wu. ``An FPGA-Optimized Architecture of Real-time Farneback Optical Flow". IEEE Annual International Symposium on Field-Programmable Custom Computing Machines (FCCM). 2020. pp. 223.
\end{sloppypar}


[5] \textbf{Zhe Pan}, Xiaohong Jiang, Jian Wu, Xiang Li. ``Hybrid XML Parser Based on Software and Hardware
Co-design". IEEE Annual International Symposium on Field-Programmable Custom Computing Machines (FCCM). 2019. pp. 325.

\section*{在投的学术论文}

[1] \textbf{Zhe Pan}, Xu Li, Shuibing He, Xuechen Zhang, Rui Wang, Yunjun Gao, Gang Chen, Xian-He Sun. ``AMBEA: Aggressive Maximal Biclique Enumeration in Large Bipartite Graph Computing ''. IEEE Transactions on Computers (TC). 2024. Under Review (Major revision). (对应第二章)

[2] \textbf{Zhe Pan}, Shuibing He, Xu Li, Xuechen Zhang, Rui Wang, Yanlong Yin, Gang Chen. ``Advanced Maximal Biclique Enumeration on GPUs Using Bitmaps''. IEEE Transactions on Parallel and Distributed Systems (TPDS). 2024. Under Review. (对应第四章)

\section*{已授权的发明专利}

[1] 姜晓红, \textbf{潘哲}, 吴健, 尹建伟, 邓水光, 李莹, 吴朝晖. 基于FPGA的XML解析器、可重构计算系统. 中国发明专利. 专利号:202010061906.X. 授权公告日:2023.07.04.

[2] 姜晓红, \textbf{潘哲}, 吴健. 基于FPGA 的稠密光流计算系统及方法. 中国发明专利. 专利号:201811600605.9. 授权公告日:2023.05.23.

[3] 姜晓红, \textbf{潘哲}, 马德, 朱国权, 郝康利. 基于牛顿迭代法的非线性激活函数计算装置. 中国发明专利. 专利号:202011090563.6. 授权公告日:2022.06.21.

\section*{在申请的发明专利}

[1] 何水兵, \textbf{潘哲}, 李旭. 一种基于候选顶点合并技术的极大二分团枚举方法. 中国发明专利. 专利号:202311387848.X. 专利公开日:2024.01.05. (对应第二章)

[2] 何水兵, \textbf{潘哲}, 李旭. 一种基于混合存储的极大二分团枚举方法. 中国发明专利. 专利号:202311385166.5. 专利公开日:2024.01.05. (对应第三章)

[3] 孙贤和, 李旭, 何水兵, \textbf{潘哲}, 陈刚. 一种GPU负载均衡的极大二分团枚举方法. 中国发明专利. 专利号:202311563374.X. 专利公开日:2024.02.06. (对应第四章)

% \ifthenelse{\equal{\BlindReview}{true}}
% {% For blind review
%     \section*{以第一作者身份发表论文}
%     \begin{enumerate}
%         \item Paper list without author information.
%         \item Paper list without author information.
%         \item Paper list without author information.
%     \end{enumerate}

%     \section*{以共同第一作者(第2位)身份发表论文}
%     \begin{enumerate}
%         \item Paper list without author information.
%         \item Paper list without author information.
%         \item Paper list without author information.
%     \end{enumerate}

%     \section*{其他论文}
%     \begin{enumerate}
%         \item Paper list without author information.
%         \item Paper list without author information.
%         \item Paper list without author information.
%     \end{enumerate}

%     \section*{发明专利}
%     \begin{enumerate}
%         \item Patent list without author information.
%         \item Patent list without author information.
%         \item Patent list without author information.
%     \end{enumerate}
% }
% {% For submission
%     \section*{期刊论文}
%     \begin{enumerate}
%         \item Paper list.
%         \item Paper list.
%         \item Paper list.
%     \end{enumerate}

%     \section*{会议报告}
%     \begin{enumerate}
%         \item Paper list.
%         \item Paper list.
%         \item Paper list.
%     \end{enumerate}

%     \section*{发明专利}
%     \begin{enumerate}
%         \item Patent list.
%         \item Patent list.
%         \item Patent list.
%     \end{enumerate}
% }