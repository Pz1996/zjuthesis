
\section{研究挑战}

尽管在极大二分团枚举领域已经有很多出色的工作,但是在处理大规模二分图时,已有方法的计算性能仍然有很大的提升空间。本节将指出主流的基于集合枚举树的枚举方法所面临的三个共性问题,并介绍解决这些问题所面临的具体挑战。

\subsection{搜索空间大,剪枝方法欠佳}

极大二分团枚举问题具有搜索空间大的特点。常见的图算法,如深度优先搜索~\cite{wiki-dfs}、广度优先搜索~\cite{wiki-bfs}、最小生成树~\cite{wiki-mst}、最短路径等算法~\cite{wiki-sssp},其搜索空间随着图的规模线性增长。而常见的图模式挖掘算法的目标子图往往只包含少量顶点~\cite{peregrine20,pangolin20,g2miner22,decomine22,khuzdul23,gamma23,Graphset23},例如三角计数问题中目标子图仅包含3个顶点~\cite{triangle18}。相比之下,极大二分团枚举问题的搜索空间更大,因为它随着二分图中顶点数量的指数级增长,并且目标子图(即极大二分团)中的顶点数量相对较多。为了应对搜索空间巨大的挑战,研究人员提出了各种优化技术,旨在减少搜索空间中产生非极大二分团的无效枚举节点,进而减少枚举时间。然而,由于搜索空间的规模庞大,现有的优化方法往往难以完全覆盖所有无效节点。具体而言,通过~\ref{subsec:ambea_exp_overall}节的实验,我们观察到现有的最新算法ooMBEA在Github数据集上需要检查并消除比极大二分团数量多26倍的产生非极大二分团的无效枚举节点。这些无效枚举节点带来大量的节点检查开销,严重降低了计算性能。因此,如何设计高效的剪枝方法来裁剪巨大搜索空间仍然是一个关键的挑战。


\subsection{计算不规则,单一结构低效}

极大二分团枚举问题具有计算不规则的特点。与其他图计算问题类似,在真实世界中,二分图的顶点邻居数量存在较大差异,导致每次计算涉及的顶点数量不同,即计算不规则~\cite{Irregularity12}。为了高效地解决这类问题,研究者们提出了多种存储结构来表示图的邻接关系。常用的存储结构包括位图~\cite{lcm04,lcmmbc07,FCA15,FCA21,FCA22}、邻接表~\cite{iMBEA14,PMBE20,ooMBE22}和哈希表~\cite{parMBE19}。不同的存储结构适用于不同场景。
例如,位图结构采用位运算,具有高效的计算能力,但在稀疏图场景下会占用更多的存储资源,因此适用于稠密小图;邻接表精确地存储每个顶点的邻居信息,在处理稀疏大图时占用较少的内存,但计算过程需要执行大量的比较运算,其运行时间与顶点个数成正比,会导致计算相对低效,因此适用于稀疏大图;哈希表具有灵活性和便于快速查找的特点,可以快速判断任意两个顶点之间的连接关系,但相比邻接表,它需要更多的存储空间并且访问方式更为随机。然而,目前的研究往往采用固定的数据结构来存储顶点的邻居信息,未能充分发挥不同数据结构的优势。因此,在枚举过程中如何动态选择合适的存储结构,发挥计算潜力,是一个关键挑战。



\subsection{负载不均匀,并行扩展性差}

极大二分团枚举问题具有负载不均匀的特点,限制了问题的并行扩展能力。为了进一步提高问题求解的效率,研究人员尝试设计并行算法来处理极大二分团枚举问题~\cite{mapreduceMBE16, MBEHe18, parMBE19}。这些算法将整个集合枚举树分解成多个子枚举树,然后利用分布式系统或多核CPU的大量计算资源来并行处理这些子树。然而,由于不同子枚举树的极大二分团的大小不同,导致计算负载之间存在较大差异。因此,即使有大量计算资源可用,计算任务的运行时间仍然受限于最耗时的负载。GPU作为一种专门用于并行计算的硬件设备,由于其内部拥有大量的计算单元,非常适合处理并行任务。然而,由于GPU和CPU在体系结构、内存层次结构以及编程模型等方面存在较大差异,现有的极大二分团枚举算法无法直接迁移到GPU系统中。尽管GPU被广泛用于加速相关的图算法,如极大团枚举~\cite{MCEGPUBitset13,MCEGPUdpp17,MCE-GPU21} 和图模式挖掘~\cite{g2miner22,SubgraphGpu22,Kclique22,stmatch22},但在GPU上进行极大二分团枚举问题仍然具有挑战性。具体来说,GPU上的极大二分团枚举问题面临着与极大二分团枚举问题类似的性能问题,许绍显等人指出GPU加速极大团枚举问题的研究极为有限~\cite{MCEreview22}。即使最新的基于GPU的极大团枚举算法GBK~\cite{MCE-GPU21} 获得较低的性能,也仅与CPU上的单线程串行算法相当。GPU上的子图枚举问题中被枚举子图通常仅包含少量顶点,而极大二分团通常包含大量顶点,因此带来更加严重的负载不均问题,导致最新的基于GPU的GPM框架G$^2$Miner~\cite{g2miner22}与GraphSet~\cite{Graphset23}中的优化无法直接解决GPU上进行极大二分团枚举面临的负载不均匀问题。因此,如何实现负载均衡并突破现有算法在并行能力方面的限制,设计基于GPU的并行极大二分团枚举方法是一个重要挑战。